\documentclass[11pt,a4paper]{article}

% --- PREAMBULE (Identique à la version précédente) ---
\usepackage[T1]{fontenc}
\usepackage[utf8]{inputenc}
\usepackage[french]{babel}
\usepackage{textcomp} % Required for \textbullet
\usepackage[a4paper,margin=2.2cm,top=2.5cm,bottom=2.5cm]{geometry}
\usepackage{fancyhdr}
\usepackage{lastpage}
\setlength{\headheight}{24pt}
\usepackage{lmodern}
\usepackage{microtype}
\usepackage{xcolor}
\usepackage{graphicx}
\usepackage{float}
\usepackage{tikz}
\usetikzlibrary{shapes,arrows,positioning}
\usepackage{array}
\usepackage{longtable}
\usepackage{booktabs}
\usepackage{enumitem}
\usepackage{multicol}
\usepackage{hyperref}
\usepackage{url}
\usepackage{titlesec}
\usepackage{titletoc}
\usepackage{listings}
\usepackage{tcolorbox}
\tcbuselibrary{most}

% --- Définitions de couleurs et commandes (Identiques) ---
\definecolor{primaryblue}{RGB}{0,102,204}
\definecolor{secondaryblue}{RGB}{51,153,255}
\definecolor{lightblue}{RGB}{230,242,255}
\definecolor{darkblue}{RGB}{0,51,102}
\definecolor{accentblue}{RGB}{102,178,255}
\definecolor{grayblue}{RGB}{108,117,125}
\definecolor{codeblue}{RGB}{0,119,187}
\definecolor{successgreen}{RGB}{40,167,69}
\definecolor{warningorange}{RGB}{255,193,7}
\definecolor{dangerred}{RGB}{220,53,69}

\hypersetup{
    colorlinks=true,
    linkcolor=primaryblue,
    filecolor=primaryblue,
    urlcolor=codeblue,
    citecolor=darkblue,
    pdftitle={Auto-Creation-VM - Guide Technique et d'Utilisation},
    pdfauthor={Projet Auto-Creation-VM},
    pdfsubject={Guide technique et d'utilisation du système multi-hyperviseur},
    pdfkeywords={VMware, Nutanix, Python, Flask, PowerShell, Virtualisation},
    bookmarksnumbered=true,
    bookmarksopen=true,
    pdfstartview=FitH
}

\pagestyle{fancy}
\fancyhf{}
\fancyhead[L]{\textcolor{primaryblue}{\textbf{Auto-Creation-VM}}}
\fancyhead[R]{\textcolor{grayblue}{\leftmark}}
\fancyfoot[L]{\textcolor{grayblue}{\today}}
\fancyfoot[C]{\textcolor{primaryblue}{\textbf{Guide Technique et d\'Utilisation}}}
\fancyfoot[R]{\textcolor{grayblue}{Page \thepage\ sur \pageref{LastPage}}}
\renewcommand{\headrulewidth}{0.5pt}
\renewcommand{\footrulewidth}{0.5pt}
\renewcommand{\headrule}{\hbox to\headwidth{\color{primaryblue}\leaders\hrule height \headrulewidth\hfill}}
\renewcommand{\footrule}{\hbox to\headwidth{\color{primaryblue}\leaders\hrule height \footrulewidth\hfill}}

\titleformat{\section}{\Large\bfseries\color{primaryblue}}{\thesection}{1em}{}
\titleformat{\subsection}{\large\bfseries\color{secondaryblue}}{\thesubsection}{1em}{}
\titleformat{\subsubsection}{\normalsize\bfseries\color{darkblue}}{\thesubsubsection}{1em}{}
\titleformat{\paragraph}{\normalsize\bfseries\color{accentblue}}{\theparagraph}{1em}{}

\setlist[itemize,1]{label=\textcolor{primaryblue}{\textbullet}}
\setlist[itemize,2]{label=\textcolor{secondaryblue}{\textendash}}
\setlist[enumerate,1]{label=\textcolor{primaryblue}{\arabic*.}}

\newtcolorbox{infobox}[1][]{
    colback=lightblue,
    colframe=primaryblue,
    fonttitle=\bfseries,
    title=Information,
    #1
}

% ... autres boîtes ...

% --- Définitions listings (Identiques) ---
\lstdefinestyle{bluecode}{
    basicstyle=\ttfamily\small\color{darkblue},
    keywordstyle=\bfseries\color{primaryblue},
    commentstyle=\itshape\color{grayblue},
    stringstyle=\color{successgreen},
    numberstyle=\tiny\color{grayblue},
    identifierstyle=\color{darkblue},
    backgroundcolor=\color{lightblue},
    frame=leftline,
    framerule=3pt,
    rulecolor=\color{primaryblue},
    showstringspaces=false,
    breaklines=true,
    breakatwhitespace=true,
    tabsize=2,
    columns=fullflexible,
    keepspaces=true,
    numbers=left,
    numbersep=10pt,
    xleftmargin=15pt,
    framexleftmargin=10pt,
    captionpos=b,
    aboveskip=10pt,
    belowskip=10pt
}

\lstdefinelanguage{JSON}{ keywords={true,false,null}, keywordstyle=\color{primaryblue}, stringstyle=\color{successgreen}, morestring=[b]" }
\lstdefinelanguage{PowerShell}{ keywords={param,begin,process,end,if,else,elseif,switch,foreach,while,do,until,try,catch,finally,throw,function,return,Write-Host,New-Item,Copy-Item,Get-ChildItem,Where-Object,Remove-Item,Join-Path,Get-Date,ConvertFrom-Json,-Action,-TemplateName,-NewVMName,-ClusterName,-VMName}, keywordstyle=\bfseries\color{primaryblue}, commentstyle=\itshape\color{grayblue}, morecomment=[l]{\#}, morestring=[b]", morestring=[b]', sensitive=false }
\lstdefinestyle{pythonblue}{style=bluecode, language=Python}
\lstdefinestyle{powershellblue}{style=bluecode, language=PowerShell}
\lstdefinestyle{bashblue}{style=bluecode, language=bash}
\lstdefinestyle{jsonblue}{style=bluecode, language=JSON}
\lstset{style=bluecode}

\newcommand{\blueemph}[1]{\textcolor{primaryblue}{\textbf{#1}}}
\newcommand{\bluecode}[1]{\textcolor{codeblue}{\texttt{#1}}}
\newcommand{\bluelink}[2]{\href{#1}{\textcolor{codeblue}{#2}}}

\setlength{\parskip}{6pt}
\setlength{\parindent}{0pt}
\renewcommand{\baselinestretch}{1.1}

\begin{document}

% --- Page de Titre (Identique) ---
\begin{titlepage}
    \centering
    \vspace*{2cm}
    {\Huge\bfseries\color{primaryblue}Auto-Creation-VM}
    \vspace{0.5cm}
    {\Large\color{secondaryblue}Système de Gestion Multi-Hyperviseur}
    \vspace{1cm}
    \textcolor{primaryblue}{\rule{\linewidth}{2pt}}
    \vspace{1cm}
    {\LARGE\bfseries\color{darkblue}Guide Technique et d'Utilisation}
    \vspace{2cm}
    \begin{tcolorbox}[
        colback=lightblue,
        colframe=primaryblue,
        width=0.8\textwidth,
        arc=5pt,
        boxrule=1pt
    ]
        \centering
        \textbf{\color{primaryblue}Technologies Supportées}
        \vspace{0.5cm}
        \begin{multicols}{2}
            \textcolor{darkblue}{\textbf{Hyperviseurs:}}
            \begin{itemize}[leftmargin=*]
                \item VMware Workstation
                \item Nutanix AHV
            \end{itemize}
            
            \textcolor{darkblue}{\textbf{Interfaces:}}
            \begin{itemize}[leftmargin=*]
                \item Interface Web (Flask)
                \item CLI Python
                \item Scripts PowerShell
                \item API REST
            \end{itemize}
        \end{multicols}
    \end{tcolorbox}
    \vfill
    \textcolor{grayblue}{\small
        \textbf{Architecture:} Modulaire \textbullet{} \textbf{Langages:} Python, PowerShell, JavaScript \\
        \textbf{Base de données:} MySQL \textbullet{} \textbf{Authentification:} JWT \textbullet{} \textbf{Tests:} Automatisés
    }
\end{titlepage}

\newpage
\tableofcontents
\newpage

% --- NOUVELLE STRUCTURE COMMENCE ICI ---

\section{Guides d'Utilisation}

\subsection{Démarrage Rapide}

\subsubsection{Installation et Configuration}
\begin{lstlisting}[language=bash]
# 1. Cloner le repository
git clone <url_du_repository>
cd Auto-Creation-VM

# 2. Installer les dépendances Python
pip install -r requirements.txt

# 3. Configurer la base de données MySQL (voir section dédiée)

# 4. Configurer les hyperviseurs
cp hypervisor_config.example.json hypervisor_config.json
# Éditer hypervisor_config.json avec vos paramètres

# 5. Démarrer l'application
python app.py
\end{lstlisting}

\subsubsection{Première Utilisation}
\begin{enumerate}
  \item Accéder à l'interface web : \bluecode{http://localhost:5000}.
  \item Créer un compte utilisateur et se connecter.
  \item Vérifier le statut des providers dans l'onglet "Paramètres".
  \item Créer votre première VM via l'onglet "Créer VM".
\end{enumerate}

\subsection{Utilisation CLI (\texttt{vm\_manager.py})}
\begin{lstlisting}[language=bash]
# Lister toutes les VMs
python vm_manager.py --list

# Créer une nouvelle VM
python vm_manager.py --create --name "cli-vm-test" --template "ubuntu-template"

# Démarrer une VM
python vm_manager.py --start --name "cli-vm-test"
\end{lstlisting}

\subsection{Utilisation PowerShell}
\begin{lstlisting}[language=powershell]
# Interface PowerShell vers vm_manager.py
.\vm_manager.ps1 -Action list

# Clonage VMware avancé
.\New-VMFromClone.ps1 -TemplateName "ubuntu-template" -NewVMName "ps-vm-test"

# Création VM Nutanix
.\New-NutanixVM.ps1 -VMName "nutanix-ps-vm" -ClusterName "cluster01"
\end{lstlisting}

\section{Explication des Fichiers du Projet}

\subsection{Fichiers Principaux}
\begin{description}
    \item[\texttt{app.py}] Application Flask principale, point d'entrée de l'API REST et de l'interface web.
    \item[\texttt{hypervisor\_manager.py}] Couche d'abstraction qui unifie les commandes pour les différents hyperviseurs.
    \item[\texttt{vm\_manager.py}] Script principal pour l'utilisation en ligne de commande (CLI).
    \item[\texttt{vm\_organizer.py}] Contient la logique pour organiser les fichiers des VMs après création.
    \item[\texttt{ip\_manager.py}] Gère l'allocation des adresses IP pour les nouvelles VMs.
    \item[\texttt{hypervisor\_config.json}] Fichier central de configuration pour les accès aux hyperviseurs.
\end{description}

\subsection{Providers d'Hyperviseurs (\texttt{hypervisor\_providers/})}
\begin{description}
    \item[\texttt{base\_provider.py}] Définit la classe de base que chaque provider doit implémenter.
    \item[\texttt{vmware\_provider.py}] Implémentation spécifique pour VMware Workstation (via \texttt{vmrun}).
    \item[\texttt{nutanix\_provider.py}] Implémentation spécifique pour Nutanix AHV (via l'API REST).
\end{description}

\subsection{Interface Web (\texttt{frontend/})}
\begin{description}
    \item[\texttt{index.html}] Page principale de l'application (tableau de bord).
    \item[\texttt{script.js}] Contient toute la logique JavaScript pour interagir avec l'API.
    \item[\texttt{style.css}] Feuille de style principale.
\end{description}

\section{Architecture Technique}

\subsection{Diagramme Simplifié}
\begin{lstlisting}[language=,basicstyle={\small\ttfamily}]
+-------------------------------------------------------------+
|                    COUCHE PRÉSENTATION                     |
|      (Interface Web, CLI Python, Scripts PowerShell)        |
+-------------------------------------------------------------+
                             |
+-------------------------------------------------------------+
|                     COUCHE MÉTIER                          |
|      (HypervisorManager, Logique d'application)            |
+-------------------------------------------------------------+
                             |
+-------------------------------------------------------------+
|                 COUCHE ABSTRACTION (Providers)             |
|      (VMwareProvider, NutanixProvider)                     |
+-------------------------------------------------------------+
                             |
+-------------------------------------------------------------+
|                  COUCHE INFRASTRUCTURE                     |
|      (VMware Workstation, Nutanix AHV, MySQL)              |
+-------------------------------------------------------------+
\end{lstlisting}

\section{API REST}

\subsection{Endpoints Principaux}
\begin{description}
    \item[\texttt{GET /api/vms}] Lister toutes les machines virtuelles.
    \item[\texttt{POST /api/vms}] Créer une nouvelle machine virtuelle.
    \item[\texttt{POST /api/vms/clone}] Cloner une machine virtuelle.
    \item[\texttt{POST /api/vms/\{name\}/start}] Démarrer une VM.
    \item[\texttt{POST /api/vms/\{name\}/stop}] Arrêter une VM.
    \item[\texttt{DELETE /api/vms/\{name\}}] Supprimer une VM.
    \item[\texttt{GET /api/providers/status}] Obtenir le statut des hyperviseurs.
\end{description}

\section{Configuration}

\subsection{Configuration \texttt{hypervisor\_config.json}}
Ce fichier est crucial. Vous devez y renseigner les informations de connexion à vos hyperviseurs.

\subsubsection{Exemple pour VMware}
\begin{lstlisting}[style=jsonblue]
"vmware": {
    "vmrun_path": "C:/Program Files (x86)/VMware/VMware Workstation/vmrun.exe",
    "templates_directory": "C:/VM_Templates",
    "default_vm_directory": "C:/VMs"
}
\end{lstlisting}

\subsubsection{Exemple pour Nutanix}
\begin{lstlisting}[style=jsonblue]
"nutanix": {
    "prism_central_ip": "10.0.0.100",
    "port": 9440,
    "username": "admin",
    "password": "password",
    "default_cluster": "cluster01"
}
\end{lstlisting}

\end{document}